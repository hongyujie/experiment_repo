\documentclass{article}
% 核心配置(Windows兼容)
\usepackage[fontset=windows]{ctex} % 中文字体
\usepackage{newtxtext} % 提供Times New Roman兼容的英文文本字体
\usepackage{geometry}
\usepackage{ulem} % 提供更好的下划线支持
\geometry{a4paper, top=2cm, bottom=2cm, left=2cm, right=2cm}
\usepackage{framed} % 用于生成大黑框
\setlength{\FrameRule}{1pt} % 大黑框边框粗细(1pt=明显黑框)
\setlength{\FrameSep}{1em} % 框内内容与边框的间距(避免拥挤)

% 正文默认五号字
\begin{document}
% 标题(居中加粗,匹配Word样式)
\begin{center}
    \bfseries \zihao{3} 基于卷积神经网络的\textbf{FashionMNIST}分类
\end{center}

% 顶部信息栏(下划线占位符,和Word一致)
{\zihao{-4} % 设置为小四号字
\noindent 考试科目:\uline{多媒体信息系统} \quad 课程编号:\rule{2cm}{0.15mm} \quad 考卷类型:学术论文报告 \\ 
\noindent 姓名:\rule{2cm}{0.15mm} \quad 学号:\rule{2cm}{0.15mm} \quad 阅卷教师:\rule{2cm}{0.15mm} \quad 成绩:\rule{2cm}{0.15mm}
}

% "答案必须写在答题纸上"提示
\vspace{1.0em} % 减小垂直间距
\noindent\centering % 使用\centering代替center环境
\textbf{(答案必须写在答题纸上)}
\vspace{-1.0em} % 使用负间距进一步减小与黑框的距离

% 大黑框区域(包裹考试要求+评分标准)
% 默认会首行缩进,使用\noindent可以取消首行缩进
\begin{framed}


这里是要写的内容











\end{framed}

\end{document}
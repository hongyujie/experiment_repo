\documentclass{article}
% 核心配置(Windows兼容)
\usepackage[fontset=windows]{ctex} % 中文字体
\usepackage{geometry}
\usepackage{ulem} % 提供更好的下划线支持
\usepackage{multicol} % 用于双栏排版
\geometry{a4paper, top=2cm, bottom=2.5cm, left=2cm, right=2cm}

% 正文默认五号字
\raggedbottom % 允许页面底部有不同高度,避免Underfull \vbox警告
\begin{document}
% 标题(居中加粗,匹配Word样式)
\begin{center}
    \bfseries \zihao{3} 宁波大学硕士研究生\textbf{2025/2026}学年第\textbf{1}学期期末考试试卷 \\
    \bfseries \zihao{3} 答题纸及评分标准
\end{center}

% 顶部信息栏(下划线占位符,和Word一致)
{\zihao{-4} % 设置为小四号字
\noindent 考试科目:\uline{多媒体信息系统} \quad 课程编号:\rule{2cm}{0.15mm} \quad 考卷类型:学术论文报告 \\ 
\noindent 姓名:\uline{洪玉杰} \quad 学号:\uline{2511100136} \quad 阅卷教师:\rule{2cm}{0.15mm} \quad 成绩:\rule{2cm}{0.15mm}
}

% "答案必须写在答题纸上"提示
\vspace{0.1em} % 减小垂直间距
\begin{center}
\textbf{(答案必须写在答题纸上)}
\end{center}
\vspace{-1.5em} % 调整与下方边框的距离

\setlength{\parindent}{2em}
\begin{center}
    \textbf{\zihao{-3} 基于深度学习的多模态行人重识别综述}
\end{center}

{\noindent \bfseries \zihao{-5} 摘\quad 要:}认识海洋、经略海洋离不开水声通信及网络技术的发展。该文对水声通信技术和水声通信网络进行综述,首先回顾了水声通信技术和水声通信网络的发展,总结了水声信道的特点。然后,对于水声通信技术中的
非相干调制技术、相干调制技术以及以应用需求为导向的新型通信技术进行陈述。随后,对于水声通信网络中数
据链路层媒介接入控制协议、网络层的路由协议和跨层设计进行分类探讨。最后,对目前水声通信及网络技术的
不足进行总结,并且对未来水声通信及网络技术的发展进行

{\noindent \bfseries \zihao{-5} 关键词:}水声通信; 水声通信网络技术; 调制解调;

\vspace{1em}

\begin{center}
    \textbf{\zihao{-3} Review of Multi-modal Pedestrian Recognition Based on Deep Learning}
\end{center}

{\noindent \bfseries \zihao{-5} Abstract:}
The UAC technology and Underwater Acoustic Communication Network(UACN) are
reviewed in this paper. Firstly, the development of underwater acoustic communication technology and
underwater acoustic communication network is reviewed, and the characteristics of underwater acoustic channel
is summarized. Then, the incoherent modulation technology, coherent modulation technology and new
communication technology oriented to application requirements in underwater acoustic communication
technology are described, and the data link layer media access control protocol, network layer routing protocol
and cross-layer design in underwater acoustic communication network are classified and discussed.

{
\noindent \bfseries \zihao{-5} Keywords:}
Underwater Acoustic Communication(UAC), 
Modulation and Demodulation, Deep Learning, Multi-modal Pedestrian Recognition

\begin{multicols}{2}
{
\noindent \bfseries \zihao{-4} 1\quad 引言}
\par\indent 近年来,随着海洋强国战略的提出,国家大力发展海洋事业,作为水下无线通信的重要组成部
分,水声通信及网络技术引起广泛关注。水声通信(Underwater Acoustic Communication, UAC)不仅在民用领域极大应用,在商业领域与军事领域等诸多方面也发挥着重要作用[1–4]。民用领域,水声通信及网络技术可用于监测水下环境和收集海洋数据。商业领域,该技术广泛用于遥控、遥测、数据回传、协同作业等。军事上,水声通信及网络技术
也发挥着巨大作用,可用于寻找水下水雷、保护港口和潜艇、监控和监视等[5–8]。随着各个海洋大国科技水平的提高,构建空天地海潜一体化网络的进程势不可挡[9]。目前,空天地海的信息传输与集成技术已相对成熟,对海洋信息的需求也愈发强烈,要提高对海洋信息的获取能力,就离不开水声通信及网络技术的发展。
水声通信技术在发送端把信息加载到声波中,通过声波将信息传输到接收端,水声通信是目前应用于水下环境中最成熟可靠的无线通信方式。陆地无线环境中电磁波通信和光通信占主导地位,但是在水下,这两种通信方式的表现不尽如人意。电磁波在水下衰减严重,且频率越高衰减越大,因此在水下,电磁波只能实现短距离的高速通信,不能满足远距离水下通信与组网的要求[10]。水下光通信通常使用蓝绿激光,这是因为蓝绿色激光(波长为470~570 nm)在水下传输时能量衰减很小,其衰减率约为0.155~0.5 dB/m[11]。水下激光通信工作频率高,传输速率可达千兆,传输时延低。但是,水下光损耗大,对水介质有很高要求,具有极强的方向性,而且海洋生物会对水下光通信造成极大干扰,建立长距离通信难度极大。声波在水下传播的速度可达1 500 m/s, 1 Hz~50 kHz的声衰减系数为10–4~10–2 dB/m[12,13],低频率、高功率的声波可在水下传播数千公里,是实现水下远程无线通信的唯一手段[14],因此,水声通信成为水下长距离通信的最佳和唯一选择。各种水下通信方式的优缺点如表1所示。随着点对点水声通信技术发展日趋成熟,以及复杂应用场景对水下信息传输技术提出更高要求,声通信网络(Underwater Acoustic Communication Network, UACN)技术得到了越来越多的关注与发展。水声通信网络与水声通信一样,在水下依靠声波传递信息。从硬件角度可分为广义和狭义两种:广义水声通信网络是由布放在水底、水中的通节点(包括固定的通信节点和装载在移动平台的通信节点)、水面浮标节点、水面移动平台、岸基通信平台以及通信卫星等构成的,其结构如图1所示;狭义水声通信网络由水下及水面部分组成,包括水底、水面以及水中的固定通信节点和自主水下潜器(Autonomous Underwater Vehicle, AUV)、水面浮标等搭载的水声通信节点构成。无论广义水声通信网络,还是狭义水声通信网络,相较于点对点水声通信,都能够增加通信覆盖范围,提升通信效率等,以适应更多、更复杂的任务场景[15]。本文中提到的水声通信网络都是指狭义水声通信网络。水声通信网络为确保能够有效而可靠地传输数据,
按照一定的网络协议运行。将网络进行分层可降低网络协议设计的复杂性,参考国际标准化组织提出的开放系统互联(Open Systems Interconnection,OSI)模型和传输控制协议/因特网互联协议(Transmission Control Protocol/Internet Protocol,TCP/IP)[16],实际应用中通常将水声通信网络层次划分为物理层、数据链路层和网络层, 图2为UACN, OSI以及TCP/IP 3种协议体系的对比。

\vspace{1em}

{\noindent \bfseries \zihao{-4} 2\quad 水声通信及网络技术的发展以及研究现状}\par
{\noindent \bfseries \zihao{5} 2.1\quad 水声通信及网络技术的发展}\par
{\noindent \bfseries \zihao{5} 2.1.1\quad 水声通信技术}\par \indent
水声通信技术的早期历史可以追溯到20世纪
50年代,经过几十年的发展,已经有了显著进步。
当前,水声通信主要向高有效性和高可靠性发展。
如图3所示,与无线电通信从1~5G的发展历程相
似,水声通信技术的发展趋势概括地说是从模拟通
信到数字通信,从非相干通信到相干通信,从单载
波通信到多载波通信,从点对点通信到网络化通信。
如今,水声通信已经发展到以全双工通信为代
表的“5G” 技术[17]。水声通信发展初期,使用的
大多是幅度调制(Amplitude Modulation, AM)和单
边带调制(Single-Side Band, SSB)水下电话传递模
拟信号。 20世纪80年代早期,水下数字频移键控
(Frequency Shift Keying, FSK)技术得到应用,
80年代后期出现相干水声通信[18]。 90年代,由于数
字信号处理(Digital Signal Processing, DSP)
芯片技术和数字通信理论的发展,一系列复杂的信
道均衡技术的实现进一步推动相干水声通信技术的
发展[19]。 20世纪90年代,美国Scripps海洋研究所
率先提出了单载波相干水声通信技术[20]。从20世纪
90年代中后期开始,以正交频分复用技术为主的多
载波相干通信技术逐渐广泛应用于水声通信[21]。进
入21世纪,多入多出(Multiple-Input MultipleOutput, MIMO)、全双工等技术的使用进一步提
升了水声通信技术性能[22]。

{\noindent \bfseries \zihao{5} 2.1.2\quad 国内外研究现状}\par \indent
在水下,声波频率越高,衰减越大,因此水声通信可用带宽和通信距离相关:远距离带宽小,速率低;近距离带宽大,速率高。本节及下节介绍国内外水声通信技术研究现状,从近距离通信和中远距离通信两方面介绍,近程通信的通信距离为0~10 km,中远程通信的距离大于10 km。国外水声通信技术开展较早,美国、日本、俄罗斯、新加坡、欧盟等国家和组织都投入了大量人员与经费进行水声通信技术研究。近程水声通信方面,文献[23]使用时间反转判决反馈均衡器和正交幅度调制(Quadrature Amplitude Modulation,QAM)技术,在海试中,以60 kbit/s的数据速率实现了3 km的可靠通信。文献[24]使用128QAM在3.6 km距离实现70 kbit/s的最大数据速率,成功将数据从水下移动无人平台传输到水面舰艇。中远程水声通信方面, 2013年,文献[25]在北冰洋采用中心频率为900 Hz的相移键控(PhaseShiftKeying, PSK)信号,在560 km的距离上实现数据速率29.6 bit/s的通信导航。将直接序列扩频与双分编码技术进行结合,实现了550km内的深海远距离通信,通信速率为6.5 bit/s。

\end{multicols}
\end{document}
\documentclass{article}
% 核心配置(Windows兼容)
\usepackage[fontset=windows]{ctex} % 中文字体
\usepackage{geometry}
\usepackage{ulem} % 提供更好的下划线支持
\geometry{a4paper, top=2cm, bottom=2cm, left=2cm, right=2cm}
\usepackage{framed} % 用于生成大黑框
\setlength{\FrameRule}{1pt} % 大黑框边框粗细(1pt=明显黑框)
\setlength{\FrameSep}{1em} % 框内内容与边框的间距(避免拥挤)

% 正文默认五号字
\begin{document}
% 标题(居中加粗,匹配Word样式)
\begin{center}
    \bfseries \zihao{3} 宁波大学硕士研究生\textbf{2025/2026}学年第\textbf{1}学期期末考试试卷 \\
    \bfseries \zihao{3} 答题纸及评分标准
\end{center}

% 顶部信息栏(下划线占位符,和Word一致)
{\zihao{-4} % 设置为小四号字
\noindent 考试科目:\uline{多媒体信息系统} \quad 课程编号:\rule{2cm}{0.15mm} \quad 考卷类型:学术论文报告 \\ 
\noindent 姓名:\uline{洪玉杰} \quad 学号:\uline{2511100136} \quad 阅卷教师:\rule{2cm}{0.15mm} \quad 成绩:\rule{2cm}{0.15mm}
}

% "答案必须写在答题纸上"提示
\vspace{1.0em} % 减小垂直间距
\begin{center}
\textbf{(答案必须写在答题纸上)}
\end{center}
\vspace{-1.0em} % 使用负间距进一步减小与黑框的距离

% 大黑框区域(包裹考试要求+评分标准)
% 默认会首行缩进,使用\noindent可以取消首行缩进
\begin{framed}

\begin{center}
    \bfseries \zihao{-3} 基于深度学习的多模态行人重识别综述 \\
    洪玉杰
\end{center}


{\noindent \bfseries \zihao{5} 摘要:}认识海洋、经略海洋离不开水声通信及网络技术的发展。该文对水声通信技术和水声通信网络(UACN)进行综述,首先回顾了水声通信技术和水声通信网络的发展,总结了水声信道的特点。然后,对于水声通信技术中的
非相干调制技术、相干调制技术以及以应用需求为导向的新型通信技术进行陈述。随后,对于水声通信网络中数
据链路层媒介接入控制协议、网络层的路由协议和跨层设计进行分类探讨。最后,对目前水声通信及网络技术的
不足进行总结,并且对未来水声通信及网络技术的发展进行











\end{framed}

\end{document}
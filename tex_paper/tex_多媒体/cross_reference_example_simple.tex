\documentclass{article}
\usepackage[UTF8]{ctex} % 中文支持
\usepackage{graphicx} % 插入图片
\usepackage{amsmath} % 数学公式
\usepackage{hyperref} % 超链接

\hypersetup{
    colorlinks=true,
    linkcolor=blue
}

\begin{document}

\title{LaTeX交叉引用简单示例}
\author{示例作者}
\date{\today}
\maketitle

\section{介绍} \label{sec:intro}
在LaTeX中,交叉引用可以帮助我们方便地引用文档中的章节、图表、公式等元素。

\section{章节引用} \label{sec:sections}
\subsection{一级章节引用}
如第\ref{sec:intro}节所述,交叉引用非常有用。

\subsection{二级章节引用} \label{subsec:sub}
在\ref{subsec:sub}中,我们将介绍如何引用二级章节。

\section{图表引用} \label{sec:figures}

\begin{figure}[htbp]
    \centering
    \includegraphics[width=0.5\linewidth]{example-image}
    \caption{一张示例图片} \label{fig:example}
\end{figure}

图\ref{fig:example}是一张示例图片。

\section{公式引用} \label{sec:equations}

\begin{equation}
    E = mc^2 \label{eq:emc2}
\end{equation}

公式\ref{eq:emc2}是著名的爱因斯坦质能方程,也可以使用\eqref{eq:emc2}来引用(带括号)。

\section{表格引用} \label{sec:tables}

\begin{table}[htbp]
    \centering
    \caption{一张示例表格} \label{tab:example}
    \begin{tabular}{|c|c|c|}
        \hline
        列1 & 列2 & 列3 \\
        \hline
        数据1 & 数据2 & 数据3 \\
        \hline
    \end{tabular}
\end{table}

表\ref{tab:example}是一张示例表格。

\section{参考文献引用} \label{sec:references}

\subsection{手动编号参考文献}

如果使用手动编号的参考文献(如您当前的论文),可以直接在正文中使用方括号加数字的格式:

水声通信与水声网络的发展与应用已有大量研究[1,2]。文献[3]对水声通信技术进行了详细综述。

\subsection{使用bibitem的参考文献}

如果使用\textbackslash thebibliography环境管理参考文献,可以使用\textbackslash cite命令引用:

水声通信技术近年来发展迅速\cite{ref:xu2009}。

\begin{thebibliography}{99}
    \bibitem{ref:einstein} Albert Einstein. Relativity: The Special and General Theory. H. Holt and Company, 1920.
    \bibitem{ref:newton} Isaac Newton. Philosophiæ Naturalis Principia Mathematica. J. Societatis Regiæ ac Typis Josephi Streater, 1687.
    \bibitem{ref:xu2009} 许肖梅. 水声通信与水声网络的发展与应用[J]. 声学技术, 2009, 28(6): 811–816.
\end{thebibliography}

\section{结论} \label{sec:conclusion}
通过本示例,我们学习了如何在LaTeX中使用\verb|\label|和\verb|\ref|命令进行交叉引用,包括章节、图表、公式、表格和参考文献的引用。

\end{document}
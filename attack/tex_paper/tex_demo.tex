\documentclass{article}
% 核心配置(Overleaf兼容)
\usepackage[fontset=noto]{ctex} % 中文字体
\usepackage{geometry}
\geometry{a4paper, top=2cm, bottom=2cm, left=2cm, right=2cm}
\usepackage{framed} % 用于生成大黑框
\setlength{\FrameRule}{1pt} % 大黑框边框粗细(1pt=明显黑框)
\setlength{\FrameSep}{1em} % 框内内容与边框的间距(避免拥挤)

\begin{document}
% 标题(居中加粗,匹配Word样式)
\begin{center}
    \bfseries \zihao{2} 宁波大学硕士研究生2025/2026学年第1学期期末考试试卷 \\
    \bfseries \zihao{2} 答题纸及评分标准
\end{center}

% 顶部信息栏(下划线占位符,和Word一致)
\noindent 考试科目:多媒体信息系统 \quad 课程编号:\rule{2cm}{0.15mm} \quad 考卷类型:学术论文报告 \\
\noindent 姓名:\rule{2cm}{0.15mm}(签名) \quad 学号:\rule{2cm}{0.15mm} \quad 阅卷教师:\rule{2cm}{0.15mm} \quad 成绩:\rule{2cm}{0.15mm}

% “答案必须写在答题纸上”提示
\vspace{0.5em}
\noindent (答案必须写在答题纸上)

% 大黑框区域(包裹考试要求+评分标准)
\begin{framed}
\noindent $\bullet$ 就目前多媒体信息系统领域的热点研究问题,撰写学术报告(包括研究性学术论文报告或综述性论文报告)(试卷总分为100分)。

\noindent $\bullet$ 格式要求:可参照《电子与信息学报》等期刊论文格式。

\noindent $\bullet$ 内容要求:(1) 研究性学术论文报告:对该领域某一问题进行理论方法研究、实验分析与讨论等。(2) 综述性论文报告:对该领域某方面问题研究进展进行综述分析,讨论相关研究现状、存在问题,给出研究展望。

\noindent $\bullet$ 学术报告篇幅:8000字以上。

\noindent $\bullet$ 学术报告的文献引用:总数在30篇以上,其中,英文文献在18篇以上。


\vspace{1em}
\noindent 评分标准:\\
切题性:20\%,理论深度:50\%,创新性:15\%,内容完整性:15\% \\
总分100分。
\end{framed}

\end{document}